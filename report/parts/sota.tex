\section{Bibliography review}

A quintic Ornstein-Uhlenbeck dynamic that jointly calibrates SPX and VIX smiles is introduced in \cite{jaber2022quintic, jaber2022joint}.\\

We wish to extend this model to capture more stylized facts (long memory, roughness, Zumbach effect).\newline
Comparison to other models (path-dependency of volatility \cite{guyon2023volatility}, rough volatility \cite{gatheral2020quadratic}).

\subsection{Stylized facts}

The equities spot and options market exhibit a lot of empirical properties. A good model should be able to capture and reproduce such stylized facts.\newline
Some statistical evidence highlighted in \cite{cont2001empirical} and \cite{LeBaron2001} gives a first set of stylized facts on asset returns:

\begin{itemize}
    \item Absence of autocorrelation,
    \item Heavy tails: The tails of the return distributions exhibit power-law-like scaling behaviour. The exponents are consistent with the existence of variances, but the existence of higher moments is not guaranteed.
    \item Multi-timescale volatility clustering,
    \item Gain/loss asymettry,
    \item Leverage effect, 
    \item etc.
    \item Gaussianity: At relatively high frequencies (less than 6 months) stock
    returns do not follow a Gaussian distribution. When looking at lower frequencies they appear closer to Gaussian, but the convergence is very slow.
\end{itemize}

On top of that, some other important effects were later discovered:

\paragraph{Volatility persistence} analyzed in \cite{comte1998long} that introduces a mean-reversion dynamic for stochastic volatility to address the long-memory behaviour.

\paragraph{Rough volatility}

\paragraph{Zumbach effect} fkjzenfgjaz\\

\paragraph{VIX pricing} 



\subsection{Models}

\paragraph{Guyon}

\paragraph{Bouchaud}



\paragraph{Gatheral}

\paragraph{Abi Jaber}


\begin{center}
\begin{tabular}{|c|c|c|c|c|c|}
    \cline{2-6}
    \multicolumn{1}{c|}{} & Black-Scholes & Heston & Gatheral & Quintic OU & LeBaron \\
    \hline
    Heavy tails & \xmark & \cmark & \cmark & \cmark & \cmark \\
    \hline
    Leverage effect & \xmark & \xmark & \cmark & \xmark & \xmark \\
    \hline
    Zumbach effect & \xmark & \cmark & \xmark & \cmark & \xmark \\
    \hline
    No Martingality & \xmark & \xmark & \xmark & \xmark & \xmark \\
    \hline
    Volatility persistence & \xmark & \xmark & \xmark & \xmark & \xmark \\
    \hline
\end{tabular}
\end{center}

The hunt for a perfect statistical model for financial markets is still going on.

